% FuseBox Report
%
% Fall 2015
%
% By: Taylor Andrews

\documentclass[11pt,twocolumn,letterpaper]{article}

% System Packages
\usepackage{epsfig}
\usepackage{float}
\usepackage{caption}
\usepackage{subcaption}
\usepackage{tabu}
\usepackage{color}
\usepackage{hyperref}
\usepackage{url}

% Local Packages
% None

% Package Options
\hypersetup{
    colorlinks,
    citecolor=black,
    filecolor=black,
    linkcolor=black,
    urlcolor=black
}

% Macros
\input{macros.tex}

% Other Options
\clubpenalty = 10000
\widowpenalty = 10000

% Variables
\newcommand{\appname}{FuseBox }
\newcommand{\appnameWOspace}{FuseBox}

\newcommand{\custos}{Tutamen }
\newcommand{\custosWOspace}{Tutamen}

% Start
\begin{document}

\title{\appname}

\author{Taylor Andrews}

\maketitle

\begin{abstract}
In the age of the cloud security is an ever evolving problem. 
Services are doing their very best to protect sensitive data. 
However, in the face of malicious hacking, it can be on the user to go the extra step to make sure
their data is safe. Safety, however, should not come at the cost of
sharing data between trusted people and devices. Take Dropbox for
example~\cite{dropbox}. 
It supports many sharing features. But traditional keys are for single
users, and can only be transfered to other machines via
copying. Thus, they do not integrate well with Dropbox. 
A more robust,
and shareable, way to encrypt files is needed. 
\appname aims to solve this problem by providing a
transparent layer of security between the end user and Dropbox
without breaking the core Dropbox use cases.
Combining authentication with encryption and decryption, \appname
expands on the basic Dropbox file system.  
%Shareability is provided using Custos, a ``Secret Storage as a Service''
%(SSaaS) platform CITATION?. Custos allows for key sharing amongst
%people and computers. 
This paper will lay out
the design and implementation of \appnameWOspace, as well as 
future goals for the project.    
\end{abstract}

\section{Introduction}
\label{sec:intro}
The idea of keeping things safe from prying eyes is nothing new.
To the layman, the intricacies of
actually protecting data online are often out of reach. This problem
becomes compounded when a single person wants to safely share data
with trusted partners, or store data across multiple devices.
Cloud-based products, like Dropbox, are built
on the ability to share and sync data. How can users protect their
files while still being able to use Dropbox as they normally would? 
\par Fortunately, part of the solution to this problem already exits.
\custosWOspace, a ``Secret Storage as a Service'' (SSaaS) platform, 
provides a flexible means of storing and
retrieving secrets, in this case, cryptographic
keys~\cite{custostrios}. 
The SSaaS model supports the features of
Dropbox that \appname needs to interact with. Leveraging \custos allows \appname to 
authenticate in conjunction with cloud based services. The user's
identity can be confirmed, and the appropriate key retrieved, on any number
of platforms. In addition, \custos provides the user with the 
flexibility to choose how they want to authenticate themselves. 
\par Encryption and decryption on cloud-based systems is inhibited by
the need to transfer and move cryptographic keys. Key sharing 
with \custos means this problem is alleviated. However, this isn't the
whole solution. The user's data still needs to be protected.
\appname aims to accomplish this as transparently as
possible. To the end user, it appears that they are just normally
interacting with Dropbox. They can read, write, and create
files. Meanwhile, client side encryption within \appname
ensures that from the perspective of Dropbox files are encrypted.
If Dropbox is somehow compromised, data that passes through \appname
will not be.    

\subsection{Goals}
\label{sec:goals}
\appname was created with two primary goals: transparency and
on-demand performance. 
\par {\bf Transparent:} Dropbox is a very convenient application to
use. Put files in, get files out. The existing desktop application
allows for seamlessly uploading and downloading files~\cite{dropboxclient}. The end user
gets to take advantage of all the nice features of Dropbox at little
to no effort on their end. \appname is built to keep this same 
seamless feel while adding additional security. Meanwhile behind the
scenes encryption, decryption, and authentication work together to
secure the data.     
\par {\bf On-Demand:} \appnameWOspace's performance is bound by
the speed of internet because it must connect to Dropbox. 
\appname does not seek to overcome this
limitation. Rather, regardless of file type or file size, the client
is designed to perform equally well. On-demand performance means that
larger files don't create additional latency because of their
size. It also means that files are only downloaded when they are requested
by the user, and local copies of files aren't retained. This different 
than the default Dropbox client which
creates local copies of all files upon installation. 

\section{Design}
\label{sec:design}
\appname is, in essence, an encrypted file system built atop
Dropbox. Its components are as follows:
\par {\bf Virtual File System:} The virtual file system (VFS) is the main component of 
\appnameWOspace. It mounts on the users home computer and handles 
all interactions with Dropbox. The VFS controls when
encryption/decryption and key storage/retrieval occur.   
\par {\bf Cryptography:} Handles securely encrypting and decrypting file data
given a secret key. Also handles authenticating users. 
\par {\bf SSaaS Key Management:} Handles key storage, key retrieval, and user
authentication so that the user may access and share their keys. 

\subsection{Virtual File System}
\label{sec:fs}
\appname is a transparent and on-demand virtual file system. The major design
decision for the VFS was when the interactions with Dropbox should
occur. 
%% \par Mounting the file system is a necessary step to connect the user
%% with \appnameWOspace. I chose to give user control over when the file
%% system is mounted and when it is not. 
\par The file system authenticates to Dropbox via an oauth token. The user must
generate this token with their account information. In order to make
\appname more usable in the long-term, this oauth token need only be
requested by the user one time. Then the file system continues to
use the specified oauth token until it expires or is
revoked by the user. 
\par Interaction with Dropbox is restricted to three simple
operations: reading Dropbox file metadata, downloading a file, 
and uploading a file. The file system must use these operations in
conjunction with normal file operations such as read and write. 
\par File metadata from Dropbox is needed within the VFS whenever it
requires the attributes of a file. For example, when the user prints
the contents of a directory. It wouldn't be enough to just obtain Dropbox
metadata when opening a file, because cases like listing directory contents
aren't opening files. The design solution is to fetch Dropbox metadata
in a file system function whose purpose is to return file
attributes. In the case of \appnameWOspace, this function is getattr. 
Noteworthy is that \appname is always fetching metadata on the fly
from Dropbox. This sacrifices offline capability, such as in the
default Dropbox client, for an on-demand approach that requires less
local memory. 
\par Once metadata is obtained in getattr, the relevant file system 
operations are read, write, open, and
release. The decision boils down to whether files are downloaded in open and
uploaded in release, or if they are downloaded and uploaded as needed
in the relevant read and write operations.
In open the VFS does not yet know how large the requested read from or
write to the specified file is. Therefore if the file is downloaded in
open it must be the entire file. Likewise, if the file is uploaded in
in release it must be the entire file since the length of any previous
write operations is unknown. 
\par Though downloading in open and uploading
in release is simpler, since the amount of the file that will actually
be used is unknown, this approach sacrifices performance. Small read
operations on large files will take longer. Furthermore, local memory
will be taken up by unnecessary sections of files.  
The current iteration of
\appname goes for the simpler solution of downloading in open and
uploading in release. Future iterations of \appname will move towards
the other more efficient and less resource-consuming method.  
\par Again, it is noteworthy that regardless where communication with
Dropbox occurs \appname will
always retain its on-demand approach of downloading and uploading
files only as needed. 

\subsection{Cryptography}
\label{sec:enc}
Encryption is part what differentiates \appname from just being a basic
Dropbox client. The selection of a cryptographic algorithm,
any additional features of encryption, and authentication were key design choices. What
algorithm adheres best to the core goals of \appnameWOspace? Based on
this criteria, I choose an AES in CTR mode~\cite{AESCTR} over AES in
CBC mode~\cite{CBC}. AES was selected because it is a standard cipher. 
\par However, \appname is not bound to any one cryptographic algorithm. 
Part of the design of \appname is
flexibility in cipher. By abstracting away the actual algorithm used,
\appname can easily update to different ciphers. The file system
simply calls encrypt or decrypt while the underlying cryptographic
implementation does the work. This ensures \appname can accommodate 
future changes and advances in cryptography. 
\par AES CTR mode is ideal for \appname because it has better
performance for random access. Since the AES in CBC mode relies on a 
feedback loop, it must always decrypt and encrypt
the whole file at a time. \appname seeks to provide consistent
performance for files large and small. CBC mode would create
too large a latency for big pieces of data.   
\par The other primary goal of \appname is transparency. To have
transparency, the end user should be aware of the encryption going on
behind the scenes when they use \appname. This means that file read
operations should never show encrypted data. The same holds true for
any write operations. Furthermore, file sizes should not indicate 
data is being encrypted. Only the unencrypted file size should be
visible to the end user. In order to
accomplish this I decided to store the unencrypted file size in the
metadata of the file. 
\par Currently, \appname does not implement any file
authentication. However, I intend to add it in the future. Authentication is
valuable because it determines all parties are who they claim to
be. Much like encryption and decryption, authentication should be
decoupled from the main VFS. That way, updates and changes 
can occur without perturbing the main \appname infrastructure. 

\subsection{SSaaS Key Management}
\label{sec:keystorage}
Key storage and retrieval is a critical feature of \appnameWOspace. 
Encryption and decryption could be accomplished on a single users
computer with a locally stored key. This, however, restricts many of the
functionalities of Dropbox. It's also less secure, given someone could
steal a computer and gain access to its local keys.
Using \custos keys can be shared, and thus 
sharing features can be utilized. Keys are also more secure as they
are protected by authentication measures. For example passwords.  
The main design choice for \appname
concerns how to integrate key storage.  
\par Metadata is required to map keys to files in \custosWOspace. This
metadata could be stored either within the file itself, or in a
separate metadata file. Also, it could either be a fixed size or a
variable size. Fixed sized metadata can be quickly discarded. Though
fixed size designs are simpler, since metadata in \appname is not all the same size,
determining a fixed size would be difficult. It also doesn't allow for
easy future expansion of metadata if necessary. 
\par Storing variable sized metadata within the file itself has its
own problems. The complexity of implementation increases. To avoid
this complexity I decided to store metadata in an separate file. This
file contains all the information needed to access \custosWOspace. It
is flexible to any future expansions of metadata. 
To preserve transparency the metadata files are hidden from the user. 

\section{Implementation}
\label{sec:implementation}
The implementation of \appname approximately follows from the key aspects
of its design. That is, the virtual file system, cryptography, and
SSaaS key management
are the primary components. Source code for \appname is available at~\cite{FuseBox}.

\subsection{Virtual File System}
\label{sec:fsimp}
The file system is built using FUSE and the Dropbox Core
API. FUSE, or File System in Userspace, allows for implementation of a
full file system inside an userspace program~\cite{fuse}. The Dropbox
API allows for communication between the host computer and Dropbox
files~\cite{dropboxcore}.
\par FUSE is an ideal backbone for an encrypted, layered file system. It allows
for non-privileged users to mount and unmount \appnameWOspace. At the
moment, \appname doesn't make full use of all the functionality FUSE
offers. For example, file permissions. This, however, could be
expanded in the future. 
\par Though the main FUSE library is C-based, I chose to use python
instead. The specific binding is
fuse-python~\cite{fuse-python}. However, fuse-python has not recently
been updated to reflect changing python standards.
In the future, \appname could increase it's longevity by moving to a FUSE library
that supports python 3 rather than python 2.
\par The Dropbox Core API v1 was chosen over the Sync API~\cite{dropboxsync} or Datastore
API~\cite{dropboxdatastore} because it provides the appropriate interactions with a single
user's Dropbox. Dropbox offers several languages for
implementation. Python was the logical choice because that language is
the one I used for FUSE. For \appnameWOspace, the python API is used within FUSE
calls. This is how doing normal file operations is able to connect
with Dropbox. For example, FUSE function {\em open} utilizes the API
function {\em get\textunderscore file}, and {\em release} utilizes {\em
  put\textunderscore file}. This reflects
the design choice of where to download and upload files.

\subsection{Cryptography}
\label{sec:encimp}
As previously discussed, \appnameWOspace's encryption and decryption
is done by AES in CTR mode. This is accomplished using the  PyCrypto
library~\cite{pycrypto}.
PyCrypto was chosen because it provides support for the less
frequently used CTR
mode. It also has important features like implementing a nonce to
randomize the counter. 
\par One downside to this library was that, despite using CTR mode,
encryption blocks had to be padded out to the AES block size. Because
of this, the VFS must be able to translate between the unencrypted and
encrypted file sizes. Without being able to translate, \appname would
lose transparency since encrypted file sizes would be visible. 
\par First, to pad out the block, \appname determines the number of
padding bytes required. Then, it converts this number directly to a
character using ASCII. That character is then appended to the file
until the correct size is reached. To reverse the padding, the last
byte of the file is examined. The character there is converted into
its ASCII numbering and that many bytes are then stripped from the
file. 
\par In the future, \appname will support file authentication. This
can be done using an message authentication code (MAC). The MAC can be
computed using a hash function, giving it the name HMAC. \appname
would implement file authentication within encryption and
decryption. The MAC would be updated in encrypted and checked in
decrypt. 

\subsection{SSaaS Key Management}
\label{sec:ksimp}
Key storage and retrieval is provided
by \custosWOspace~\cite{Pytutamen}. \appname interacts with two aspects
of \custosWOspace, collections and secrets. In the current version
of \appnameWOspace, collections and secrets are a one-to-one
relationship with files. Thus, every files stores metadata containing
both a collection uuid and a secret uuid, e.g. one key per file. It also contains, as
mentioned above, the encrypted file size. See example metadata below:
\begin{center}
\noindent\texttt{\{``cuuid'':<cuuid>, ``suuid'':<suuid>, ``size'':0\}}
\hspace{\columnwidth}
\end{center}
\par One key per file allows for sharing to be as granular as
possible. That way the user can select exactly what they would like to share.
At the moment, the one-to-one relationship is simple. In the future, \appname
could be expanded to have multiple keys per file. This allows for
finer user control as one key could be more strict with it's
authentication measures and another less so.   

\section{Conclusion}
\label{sec:conclusion}
\appname provides value by being an encrypted file
system that allows for secure sharing between people and devices.
Users don't have to sacrifice security to take advantage of Dropbox. 
\par However, \appname is still incomplete. In the future, optimizations
could be made to increase consistency of performance. These include
file streaming for random access. The libraries and APIs \appname is
built on could afford to be updated. Dropbox, for example, has recently released an
updated version of their API. I look forward to continuing and
improving the \appname project.    

\bibliographystyle{acm}
\bibliography{refs}

\end{document}
